\section{Особенности языка Си}

Основные концепции языка Си
\begin{itemize}
  \item это язык сравнительно «низкого» уровня;
  \item это «маленький» язык c однопроходным компилятором;
  \item предполагает, что программист знает, что делает.
\end{itemize}

Как заметил пользователь Habr-а \textit{lesha\_penguin}:

\begin{displayquote}
  \ldots язык C -- это язык, практически не мешающий программисту делать то, что он решил делать. В нем нет ``няньки'', которая ходит за тобой, и пытается тебе ``мешать делать глупости''. Все равно от всех глупостей не застрахуешь. \\
  \ldots \\
  Язык C -- наверное единственный язык из высокоуровневых, который вам дает практически неограниченную свободу. Причем плата за эту практически полную свободу смешная -- всего-навсего ответственность!
\end{displayquote}

\subsection{Отличия языка Си от Python}

\begin{itemize}
  \item Python -- интерпретируемый язык, C -- компилируемый язык.
  \item Python -- язык с динамической типизацией, C -- язык со статической типизацией.
  \item Python -- автоматическое управление памятью, C -- ручное управление памятью.
\end{itemize}

\subsection{Пример программы на языке Си}

Программа ``Hello, world!'' на языке Си:

\begin{minted}{c}
  /*
   * Программа на языке Си
   */
  #inclide<stdio.h>

  int main(void)
  {
    printf("Hello, world!\n");
    return 0;
  }
\end{minted}

\subsection{Выполнение программы из примера}

Для начала программу требуется \textit{скомпилировать}:

\begin{minted}{bash}
  gcc -std=c99 -Wall -Werror -o hello.exe hello.c
\end{minted}

Здесь:
\begin{itemize}
  \item \texttt{gcc} -- название компилятора;
  \item \texttt{-std=c99} -- испольуемый стандарт;
  \item \texttt{-Wall} -- включение всех предупреждений;
  \item \texttt{-Werror} -- трактовать предупреждения как ошибки;
  \item \texttt{-o hello.exe} -- имя исполняемого файла;
  \item \texttt{hello.c} -- что компилировать.
\end{itemize}

\subsection{Выводы}

\begin{itemize}
  \item Программа на языке Си состоит из одной и более функций.
  \item Среди этих функций обязательно должна быть функция с именем main, с которой начнётся выполнение программы.
  \item Значение, которое возвращает функция main, говорит об успешности работы программы.
  \item Если в программе используется какая-либо функция из стандартной библиотеки, компилятору необходимо сообщить ее описание, подключив соответствующий заголовочный файл.
  \item Игнорировать предупреждения компилятора нельзя.
\end{itemize}



