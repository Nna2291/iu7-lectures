\section{Переменная в языке Си}

\begin{definition}
  Переменной называется именованный участок памяти, обладающий некоторым типом.
\end{definition}

В Си \textit{переменная} перед использованием должна быть \textit{определена}.
\begin{minted}{c}
  int x;    // Определение
  x = 42;
\end{minted}

\begin{note}
  Переменная в языке Python сильно отличаются от переменных в языке Си!
\end{note}

\subsection{Выбор типа переменной}

Выбор правильного типа для переменной очень важен, потому что тип определяет
\begin{itemize}
  \item как переменная хранится;
  \item какие значения может принимать;
  \item какие операции могут быть выполнены над переменной.
\end{itemize}

Например, целочисленный тип \texttt{int} может принимать ограниченное количество значений (от $-2^{15}+1$ до $+2^{15}-1$). Тип числа с плавающей точкой (ЧПТ) может принимать значительно больший диапазон значний, а также дробные числа. Однако операции с такими числами выполняются дольше.

\subsubsection{Пример использования переменных}

\begin{minted}{c}
#include<stdio.h>

int main(void)
{
  int a, b, c;

  printf("Enter two integers:\n");
  scanf("%d %d", &a, &b);

  c = a + b;

  printf("a + b = %d\n", c);

  return 0;
}
\end{minted}

\subsection{Выводы}

\begin{itemize}
  \item Программы обрабатывают данные, которые представлены переменными.
  \item Язык Си это язык со статической типизацией. Прежде чем использовать переменную, ее нужно описать.
  \item Описание переменной состоит из указания типа переменной и ее имени.
  \item Выбор правильного типа для переменной очень важен.
  \item Для вывода значений переменных обычно используется функция printf, работа которой управляется строкой форматирования.
  \item Для ввода значения переменных обычно используется функция scanf, работа которой управляется строкой форматирования.
  \item Алгоритм работы функции scanf основана на «сопоставлении с образцом».
  \item Функция scanf должна изменять значения переменных, поэтому в качестве параметров получает адреса этих переменных.
\end{itemize}

