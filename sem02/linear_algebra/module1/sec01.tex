\section{Линейные пространства}

\begin{definition}[Линейное пространство]
  Линейное пространство $\mathcal{L}$ над множеством значений $mathcal{P}$ (элементы будем называть векторами), для которого определены операции сложения и умножеения на скаляр, а также верно:
    \begin{enumerate}
        \item $\forall x, y \in \mathcal{L} \quad x + y = y + x$
        \item $\forall x, y \quad (x + y) + z = x + (y + z)$
        \item $\exists 0 : \forall x \in \mathcal{L} x + 0 = x$
        \item $\forall x \in \mathcal{L} \exists y : x + y$ -- существование противоположного вектора $(-x)$
        \item $\forall x \in \mathcal{L} \quad (\alpha \beta) x = \alpha (\beta x)$
        \item $\forall x \quad 1 x = x$ 
        \item $(\alpha + \beta) x = \alpha x + \beta x$
        \item $\alpha(x + y) = \alpha x + \alpha y$ 
    \end{enumerate}
\end{definition}

В рамках курса считаем линейное пространство над элементами множества $\R$. 

Примерами линейного пространства могут быть:
\begin{enumerate}
    \item Множество свободных векторов
    \item $n$-мерное пространство ($R^n$) 
    \item Множество непрерывных функция на отрезке
    \item Множество матриц одинакового размера
    \item Множество многочленов степени $n$
    \item и т.п.
\end{enumerate}

\subsubsection{Свойства линейных пространств}

\begin{property}
    Нулевый элемент единственен.
\end{property}
\begin{proof}
    Пусть существуют два нулевых элемента: $\vec{0_1}$ и $\vec{0_2}$. Тогда: \[
        \vec{0_1} = \vec{0_1} + \vec{0_2} = \vec{0_2} + \vec{0_1} = \vec{0_2}
    \]
\end{proof}

\begin{property}
    Для каждого элемента противоположный единственный.
\end{property}
\begin{proof}
    Пусть существую два противоположных элемента для $x$: $\vec{y_1}$ и $\vec{y_2}$. Тогда:
    \begin{align*}
        x + \vec{y_1} &= 0 \\
        x + \vec{y_2} &= 0 \\
        x + \vec{y_1} &= x + \vec{y_2} \\
        \vec{y_1} &= \vec{y_2}
    \end{align*}
\end{proof}

\begin{property}
    \[
        0 \cdot x = x
    \]
\end{property}
\begin{proof}
    \begin{gather*}
        0 x = 0 x + 0 = (0 + 1) x + (-x) = 0
    \end{gather*} 
\end{proof}

\begin{property}
    \[
        (-1) \cdot x = (- x)
    \]
\end{property}
\begin{proof}
    \begin{gather*}
        (-1) \cdot x + x = (1 - 1) x = 0 = x + -x \\
        \implies (-1) \cdot x = -x
    \end{gather*}
\end{proof}

\begin{property}
    Уравнение \[
        \forall x, y \in \mathcal{L} \quad x + a = y
    \]
    имеет решение и притом единственное.
\end{property}
\begin{proof}
    Пусть: \[
        a = y + (-x)
    \]
    Тогда подставляя в изначальное уравнение получаем тождество.
\end{proof}

\subsection{Линейная зависимость и независимость векторов}

Пусть есть некоторый набор векторов $\vec{x_1}, \ldots, \vec{x_k} \in \mathcal{L}$.

\begin{definition}[Линейная комбинация]
    Линейной комбинацией называется выражение вида: \[
        \lambda_1 \vec{x_1} + \lambda_2 \vec{x_2} + \ldots + \lambda_k + \vec{x_k}
    \]
\end{definition}

\begin{definition}[Тривиальная линейная комбинация]
    Линейная комбинация называется \textit{тривиальной}, если все коэффициенты равны нулю.
\end{definition}

\begin{definition}[Нетривиальная линейная комбинация]
    Линейная комбинация называется \textit{нетривиальной}, если хотя бы один коеффициент не равен нулю.
\end{definition}

\begin{definition}[Линейно зависимая комбинацию]
    Система векторов называется \textit{линейно-зависимой}, если существует нетривиальная линейная комбинация, равная нулевому вектору. 
\end{definition}

\begin{theorem}
    Чтобы система была линейно зависима, необходимо и достаточно, чтобы любой вектор вектор линейно выражался через остальные.
\end{theorem}

\begin{property}
    Если в системе векторов существует нулевой вектор, то такая система линейно зависима.
\end{property}

\begin{property}
    Если система векторов содержит линейно зависимую подсистему, то система тоже линейно зависима.
\end{property}

\begin{property}
    Если система векторов линейно независима, то и любая ее подсистема тоже линейно независима.
\end{property}

\begin{property}
    Если векторы $x_1, \ldots, x_n$ линейного пространства $\mathcal{L}$ линейно независимы и вектор $y \in \mathcal{L}$ не является их линейной комбинацией, то расширенная система векторов $x_1, \ldots , x_n, y$ является линейно независимой.
\end{property}

\subsection{Базис, размерность пространства}

\begin{definition}[Базис]
    \textit{Базисом} линейного пространства $\mathcal{L}$ называют любую упорядоченную систему векторов, для которой выполнены два условия:
    \begin{enumerate}
        \item эта система векторов линейно независима;
        \item каждый вектор в линейном пространстве может быть представлен в виде линейной комбинации векторов этой системы.
    \end{enumerate}
\end{definition}

\begin{definition}
    Коэффициенты разложения вектора по базису линейного пространства, записанные в соответствии с порядком векторов в базисе, называют \textit{координатами вектора в этом базисе}.
\end{definition}

\begin{theorem}[О единственности разложения]
    Разложение по базису \textit{единственно}.
\end{theorem}

\begin{definition}[Конечномерное пространство]
    Пространство называется \textit{конечномерное}, если сущестсвует базис конечного числа векторов.
\end{definition}

\begin{definition}[Бесконечномерное пространство]
    Пространство называется \textit{бесконечномерное}, если не сущестсвует базис конечного числа векторов.
\end{definition}

\begin{theorem}
    Если $\mathcal{L}$ -- конечномерное пространство, тогда все базисы состоят из конечного числа векторов.
\end{theorem}

\begin{definition}[Размерность линейного пространства]
    Максимальное количество линейно независимых векторов в данном линейном пространстве называют \textit{размерностью линейного пространства}. \[
        \mathrm{dim} (\mathcal{L}) = n
    \]
\end{definition}

Линейная зависимость (независимость) равносильна линейной зависимости (независимости) столбцов координат в том же базисе.

\subsection{Преобразование координат вектора при замене базиса}

Пусть в $n$-мерном пространстве $\mathcal{L}$ заданы два базиса:  \[
  b = \left( b_1, \ldots b_n \right) \quad c = \left( c_1, \ldots c_n \right) 
\] 

Любой вектор мжно разложить по базису $b$. А значит любой вектор из базиса  $c$ может быть представлен как:  \[
  c_i = \lambda_{1i} b_1 + \ldots + \lambda_{ni} b_n,
  \quad i = \overline{1, n}
\] 

Запишем в матричном виде: \[
  c_i = b
  \begin{pmatrix}
    a_{1i} \\ \ldots \\ a_{ni}
  \end{pmatrix},
  \quad i = \overline{1, n}
\] 

Или \[
  c = bU \quad U = 
  \begin{pmatrix}
    a_{11} & \ldots & a_{1n} \\
    \ldots \\
    a_{n1} & \ldots & a_{nn}
  \end{pmatrix}
  \tag{1.11}
\] 

\begin{definition}[Матрица перехода]
  Матрицу $U$ (1.11) называют \textit{матрицей перехода} от старого базиса $b$ к новому базису ,$c$.
\end{definition}

\begin{property}[1]
  Матрица перехода невырождена и всегда имеет обратную.
\end{property}

\begin{property}[2]
    Если в $n$-мерном линейном пространстве задан базис $b$, то для любой невырожденной квадратной матрицы $U$ порядка $n$ существует такой базис $c$ в этом линейном пространстве, что $U$ будет матрицей перехода от базиса $b$ к базису $c$.
\end{property}

\begin{property}[3]
  Если $U$ -- матрица перехода от старого базиса $b$ к новому базису c линейного пространства, то $U^{-1}$ -- матрица перехода от базиса $c$ к базису $b$.
\end{property}

\begin{property}[4]
  Если в линейном пространстве заданы базисы $b$, $c$ и $d$, причем $U$ -- матрица перехода от базиса $b$ к базису $c$, a $V$ -- матрица перехода от базиса $c$ к базису $d$, то произведение этих матриц $UV$ -- матрица перехода от базиса $b$ к базису $d$.
\end{property}

