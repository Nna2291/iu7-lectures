\section{Линейные пространства}

\begin{definition}[Линейное пространство]
    Линейное пространство $\mathbb{L}$ над множеством значений $P$ (элементы будем называть векторами), для которого определены операции сложения и умножеения на скаляр, а также верно:
    \begin{enumerate}
        \item $\forall \vec{x}, \vec{y} \in \mathbb{L} \quad \vec{x} + \vec{y} = \vec{y} + \vec{x}$
        \item $\forall \vec{x}, \vec{y} \quad (\vec{x} + \vec{y}) + \vec{z} = \vec{x} + (\vec{y} + \vec{z})$
        \item $\exists \vec{0} : \forall \vec{x} \in \mathbb{L} \vec{x} + \vec{0} = \vec{x}$
        \item $\forall \vec{x} \in \mathbb{L} \exists \vec{y} : \vec{x} + \vec{y}$ -- существование противоположного вектора $(-\vec{x})$
        \item $\forall \vec{x} \in \mathbb{L} \quad (\alpha \beta) \vec{x} = \alpha (\beta \vec{x})$
        \item $\forall \vec{x} \quad 1 \vec{x} = \vec{x}$ 
        \item $(\alpha + \beta) \vec{x} = \alpha \vec{x} + \beta \vec{x}$
        \item $\alpha(\vec{x} + \vec{y}) = \alpha \vec{x} + \alpha \vec{y}$ 
    \end{enumerate}
\end{definition}

В рамках курса считаем линейное пространство над элементами множества $\R$. 

Примерами линейного пространства могут быть:
\begin{enumerate}
    \item Множество свободных векторов
    \item $n$-мерное пространство ($R^n$) 
    \item Множество непрерывных функция на отрезке
    \item Множество матриц одинакового размера
    \item Множество многочленов степени $n$
    \item и т.п.
\end{enumerate}

\subsubsection{Свойства линейных пространств}

\begin{property}
    Нулевый элемент единственен.
\end{property}
\begin{proof}
    Пусть существуют два нулевых элемента: $\vec{0_1}$ и $\vec{0_2}$. Тогда: \[
        \vec{0_1} = \vec{0_1} + \vec{0_2} = \vec{0_2} + \vec{0_1} = \vec{0_2}
    \]
\end{proof}

\begin{property}
    Для каждого элемента противоположный единственный.
\end{property}
\begin{proof}
    Пусть существую два противоположных элемента для $\vec{x}$: $\vec{y_1}$ и $\vec{y_2}$. Тогда:
    \begin{align*}
        \vec{x} + \vec{y_1} &= \vec{0} \\
        \vec{x} + \vec{y_2} &= \vec{0} \\
        \vec{x} + \vec{y_1} &= \vec{x} + \vec{y_2} \\
        \vec{y_1} &= \vec{y_2}
    \end{align*}
\end{proof}

\begin{property}
    \[
        0 \cdot \vec{x} = \vec{x}
    \]
\end{property}
\begin{proof}
    \begin{gather*}
        0 \vec{x} = 0 \vec{x} + \vec{0} = (0 + 1) \vec{x} + (-\vec{x}) = \vec{0}
    \end{gather*} 
\end{proof}

\begin{property}
    \[
        (-1) \cdot \vec{x} = (- \vec{x})
    \]
\end{property}
\begin{proof}
    \begin{gather*}
        (-1) \cdot \vec{x} + \vec{x} = (1 - 1) \vec{x} = \vec{0} = \vec{x} + -\vec{x} \\
        \implies (-1) \cdot \vec{x} = -\vec{x}
    \end{gather*}
\end{proof}

\begin{property}
    Уравнение \[
        \forall \vec{x}, \vec{y} \in \mathbb{L} \quad \vec{x} + \vec{a} = \vec{y}
    \]
    имеет решение и притом единственное.
\end{property}
\begin{proof}
    Пусть: \[
        \vec{a} = \vec{y} + (-\vec{x})
    \]
    Тогда подставляя в изначальное уравнение получаем тождество.
\end{proof}

\subsection{Линейная зависимость и независимость векторов}

Пусть есть некоторый набор векторов $\vec{x_1}, \ldots, \vec{x_k} \in \mathbb{L}$.

\begin{definition}[Линейная комбинация]
    Линейной комбинацией называется выражение вида: \[
        \lambda_1 \vec{x_1} + \lambda_2 \vec{x_2} + \ldots + \lambda_k + \vec{x_k}
    \]
\end{definition}

\begin{definition}[Тривиальная линейная комбинация]
    Линейная комбинация называется \textit{тривиальной}, если все коэффициенты равны нулю.
\end{definition}

\begin{definition}[Нетривиальная линейная комбинация]
    Линейная комбинация называется \textit{нетривиальной}, если хотя бы один коеффициент не равен нулю.
\end{definition}

\begin{definition}[Линейно зависимая комбинацию]
    Система векторов называется \textit{линейно-зависимой}, если существует нетривиальная линейная комбинация, равная нулевому вектору. 
\end{definition}

\begin{theorem}
    Чтобы система была линейно зависима, необходимо и достаточно, чтобы любой вектор вектор линейно выражался через остальные.
\end{theorem}

\begin{property}
    Если в системе векторов существует нулевой вектор, то такая система линейно зависима.
\end{property}

\begin{property}
    Если система векторов содержит линейно зависимую подсистему, то система тоже линейно зависима.
\end{property}

\begin{property}
    Если система векторов линейно независима, то и любая ее подсистема тоже линейно независима.
\end{property}

\begin{property}
    Если векторы $x_1, \ldots, x_n$ линейного пространства $\mathbb{L}$ линейно независимы и вектор $y \in \mathbb{L}$ не является их линейной комбинацией, то расширенная система векторов $x_1, \ldots , x_n, y$ является линейно независимой.
\end{property}

\subsection{Базис, размерность пространства}

\begin{definition}[Базис]
    \textit{Базисом} линейного пространства $\mathbb{L}$ называют любую упорядоченную систему векторов, для которой выполнены два условия:
    \begin{enumerate}
        \item эта система векторов линейно независима;
        \item каждый вектор в линейном пространстве может быть представлен в виде линейной комбинации векторов этой системы.
    \end{enumerate}
\end{definition}

\begin{definition}
    Коэффициенты разложения вектора по базису линейного пространства, записанные в соответствии с порядком векторов в базисе, называют \textit{координатами вектора в этом базисе}.
\end{definition}

\begin{theorem}[О единственности разложения]
    Разложение по базису \textit{единственно}.
\end{theorem}

\begin{definition}[Конечномерное пространство]
    Пространство называется \textit{конечномерное}, если сущестсвует базис конечного числа векторов.
\end{definition}

\begin{definition}[Бесконечномерное пространство]
    Пространство называется \textit{бесконечномерное}, если не сущестсвует базис конечного числа векторов.
\end{definition}

\begin{theorem}
    Если $\mathbb{L}$ -- конечномерное пространство, тогда все базисы состоят из конечного числа векторов.
\end{theorem}

\begin{definition}[Размерность линейного пространства]
    Максимальное количество линейно независимых векторов в данном линейном пространстве называют \textit{размерностью линейного пространства}. \[
        \mathrm{dim} (\mathbb{L}) = n
    \]
\end{definition}

Линейная зависимость (независимость) равносильна линейной зависимости (независимости) столбцов координат в том же базисе.
