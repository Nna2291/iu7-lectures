\section{Введение}

\subsection{Базовые понятия}

\begin{definition}[Материальная точка]
    Тело, размерами которого в рамках рассматриваемой задачи можно пренебречь, называется \textit{материальной точкой}.
\end{definition}

\begin{definition}[Твёрдое тело]
    Тело, деформацией которого в рамках рассматриваемой задачи можно пренебречь, называется \textit{твёрдым телом}.
\end{definition}

\begin{definition}[Радиус-вектор]
    \textit{Радиус-вектором} называется вектор, проведённый из начала координат в данную точку.
\end{definition}

\subsection{Кинематика материальной точки}

\begin{definition}[Траектория]
    Линия, которую описывает Материальная точка при своём движении, называется \textit{траекторией}.
\end{definition}

\begin{definition}[Путь]
    Расстояние, отсчитанное вдоль траекторией, представляет из себя \textit{путь}. 
\end{definition}

\begin{definition}[Перемещение]
    Вектор, соединяющий две точки траектории, называется \textit{перемещением}. 
\end{definition}

\begin{figure}[ht]
    \centering
    \begin{tikzpicture}[domain=-4:4]
        % Рандомный график, интересно выглядит, да ведь?
        \draw[->, color=orange] plot (\x,{\x * sin((\x + 2.5) r)}) node[right] {Траектория};
        \draw[->, color=NavyBlue!] (-4, 4) -- (4, {4*sin((4 + 2.5) r)}) node[midway, above, sloped] {Перемещение};
    \end{tikzpicture}
    \caption{Траекторией и перемещение}
\end{figure}

Любое движение тела может быть описано как композиция:
\begin{itemize}
    \item Поступательного
    \item Вращательного
\end{itemize}

\begin{definition}[Поступательное движение]
    Движение, при котором любая прямая, связанная с движущимся телом, остаётся параллельной самой себе, называется \textit{поступательным движением}.
\end{definition}

\begin{definition}[Вращательное движение]
    Движение, при котором все точки тела движутся по окружностям, центры которых лежат на одной и той же прямой, называемой \textit{осью вращения}, называется \textit{вращательным движением}. 
\end{definition}

\subsubsection{Скорость}

\begin{definition}[Скорость]
    \textit{Скоростью} называется предел: \[
        \vec{v} = \lim_{\Delta t \to 0} \frac{\Delta r}{\Delta t} 
    \]
\end{definition}

Иначе можно записать так: \[
    \vec{v} = \frac{dr}{dt}
\]
